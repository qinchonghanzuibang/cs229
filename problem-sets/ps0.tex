\documentclass[11pt]{article}
\usepackage{fullpage}
\usepackage{amsmath}
\usepackage{amssymb}
\usepackage{setspace}
\usepackage{color}
\linespread{1.3}

\newcommand{\R}{\mathbb{R}}
\newcommand{\N}{\mathbb{N}}

\title{\bf{Problem Set 0}}
\author{Jonathan Qin}
\begin{document}
\maketitle


\section{Gradients and Hessians}
\begin{enumerate}


\item[(a)]
Let $f(x) = \frac{1}{2}x^{T}Ax + b^{T}x$, where $A$ is a symmetric matrix and $b \in \R^{n}$ is a vector. What is $\nabla f(x)$?

\color{blue}
$\nabla f(x) = \nabla(\frac{1}{2}x^{T}Ax) + \nabla (b^{T}).$
Let $m(x) = x^{T}Ax$, $n(x) = b^{T}x$
\begin{equation*}
\begin{aligned}
		m(x) &= \sum_{i=1}^n \sum_{j=1}^n x_{i}A_{ij}x_{j}\\
		n(x) &= \sum_{i=1}^n b_i x_i
\end{aligned}
\end{equation*}
\begin{equation*}
\begin{aligned}
	\frac{\partial}{\partial x_{k}} (x^{T}Ax) &= \frac{\partial}{\partial x_{k}} (\sum_{i=1}^n \sum_{j=1}^n x_{i}A_{ij}x_{j}) \\
     &= \sum_{i=1}^n \sum_{j=1}^n (\frac{\partial x_{i}}{\partial x_{k}}A_{ij}x_{j} + x_{i}A_{ij} \frac{\partial x_{j}}{x_{k}}) \\
     &= \sum_{j=1}^n A_{kj}{x_j} + \sum_{i=1}^n x_{i}A_{ik} \: (\frac{\partial x_i}{\partial x_k} = \delta_{ik} \text{,  Kronecker delta, is 1 if $i = k$, 0 otherwise)} \\
     &= 2 \sum_{i = 1}^n A_{ki}{x_i} \: (A \text { is symmetric, } A_{ij} = A_{ji}) \\
     &= 2 Ax \\
	\frac{\partial} {\partial x_k}(b^{T}x) &= \frac{\partial} {\partial x_k} (\sum_{i=1}^n b_i x_i) \\
	&= b
\end{aligned}
\end{equation*}
Hence, $\nabla f(x) = \frac{1}{2} \cdot 2Ax + b = Ax+b$
\color{black}

\item[(b)]
Let $f(x) = g(h(x))$,where $g:\R \rightarrow \R$ is differentiable and $h: \R ^{n} \rightarrow \R$ is differentiable. What is $\nabla f(x)$?

\color{blue}
Chain rule: 
\begin{equation*}
	\begin{aligned}
		\frac{\partial g(h(x))}{\partial x_i} &= \frac{\partial g(h(x))}{\partial h(x)} \frac{\partial h(x)}{\partial x_i} \\
&= g'(h(x)) \frac{\partial h(x)}{\partial x_i} 
\nabla f(x) &= \nabla g(h(x)) = g'(h(x)) \nabla h(x)
	\end{aligned}
\end{equation*}

\color{black}

\item[(c)]
Let $f(x) = \frac{1}{2}x^{T}Ax + b^{T}x$, where $A$ is symmetric and $b \in \R^{n}$ is a vector. what is $\nabla ^{2} f(x)?$

\color{blue}
\begin{equation*}
\begin{aligned}
	\nabla ^2 f(x) &= \nabla(Ax+b)\\
	&= [\frac{\partial}{\partial x_1}f(x) \frac{\partial}{\partial x_2}f(x) \cdots \frac{\partial}{\partial x_n}f(x)] \\
	&= \begin{bmatrix}
A_{11} & A_{12} & \dots & A_{1n} \\
A_{21} & A_{22} & \dots & A_{2n} \\
\vdots & \vdots & \ddots & \vdots \\
A_{n1} & A_{n2} & \dots & A_{nn}
\end{bmatrix} = A
\end{aligned}	
\end{equation*}

\color{black}

\item[(d)]
Let $f(x) = g(a^{T}x)$, where $g:\R \rightarrow \R$ is continuously differentiable and $a \in \R^{n}$ is a vector. What are $\nabla f(x)$ and $\nabla ^{2} f(x)$?

\color{blue}
\begin{equation*}
	\begin{aligned}
		\nabla f(x) &= \nabla g(a^T x) \\
		&= g'(a^T x) \cdot \nabla (a^T x) \; (a^T x = \sum_{i=1}^n a_i x_i)    \\
		&= g'(a^T x) \cdot a
	\end{aligned}
\end{equation*}

\begin{equation*}
	\begin{aligned}
\frac{\partial^2 g(h(x))}{\partial x_i \partial x_j} &= \frac{\partial^2 g(h(x))}{\partial (h(x))^2} \frac{\partial h(x)}{\partial x_i} \frac{\partial h(x)}{\partial x_j} = g''(h(x))\frac{\partial h(x)}{\partial x_i}\frac{\partial h(x)}{\partial x_j} \\
\frac{\partial^2 g(a^T x)}{\partial x_i \partial x_j} &= g''(a^T x)\frac{\partial (a^T x)}{\partial x_i}\frac{\partial (a^T x)}{\partial x_j} = g''(a^T x)a_i a_j \\
\nabla^2 f(x) &= \nabla^2 g(a^T x) = g''(a^T x) 
\begin{bmatrix}
a_1 a_1 & a_1 a_2 & \dots & a_1 a_n \\
a_2 a_1 & a_2 a_2 & \dots & a_2 a_n \\
\vdots & \vdots & \ddots & \vdots \\
a_n a_1 & a_n a_2 & \dots & a_n a_n
\end{bmatrix} = g''(a^T x)aa^T
	\end{aligned}
\end{equation*}
\color{black}

\end{enumerate}

\section{Positive definite matrices}

\begin{enumerate}

	\item[(a)] 
		Let $z \in \R ^n$ be an n-vector. Show that $A = zz^T$ is positive semidefinite.
		\color{blue}
		\begin{flalign*}
					&A^T = (zz^T)^T = (z^T)^Tz^T = zz^T = A&\\
			&x^Tzz^Tx = x^Tz(x^Tz)^T=(x^Tz)^2>=0 	&
		\end{flalign*}
		
		\color{black}
		
	\item[(b)] 
		Let $z \in \R ^n$ be a non-zero n-vector. Let $A = zz^T$. What is the null-space of $A$? What is the rank of $A$?
		\color{blue}
		\begin{flalign*}
			&Ax = 0  \rightarrow zz^Tx=0     &\\
			&\text{Since z is a non-zero vector}, z^Tx=0, Null(A) = \{ x \in \R^n : x^Tz = 0 \} &\\
			&Rank(A) = 1 \text{. It is the number of independent columns, which in this case, all columns are}\\
			&\text{a multiple of $z$.}&
		\end{flalign*}
		
		\color{black}
		
	\item[(c)] 
		Let $A \in \R ^{n \times n}$ be positive semidefinite and $B \in \R ^{m \times n}$ be arbitrary, where $m, n \in \N$. Is $BAB^T $ PSD? If so, prove it. If not, give a counterexample with explicit $A, B$.
		\color{blue}
		\begin{flalign*}
			&Proof:&\\ 
			&(BAB^T)^T = (B^T)^TA^TB^T = BAB^T (\text{since $A$ is symmetric})&\\
			&x^TBAB^Tx = (x^TB) A (x^TB)^T \geq 0&
		\end{flalign*}
			
		\color{black}
		
\end{enumerate}

\section{Eigenvectors, eigenvalues, and the spectral theorem}

\begin{enumerate}
	\item[(a)]
Suppose that the matrix $A \in \mathbb{R}^{n \times n}$ is diagonalizable, that is, $A = T \Lambda T^{-1}$ for an invertible matrix $T \in \mathbb{R}^{n \times n}$, where $\Lambda = \text{diag}(\lambda_1, \ldots, \lambda_n)$ is diagonal. Use the notation $t^{(i)}$ for the columns of $T$, so that $T = \left[ t^{(1)} \ \cdots \ t^{(n)} \right]$, where $t^{(i)} \in \mathbb{R}^n$. Show that $A t^{(i)} = \lambda_i t^{(i)}$, so that the eigenvalues/eigenvector pairs of $A$ are $(t^{(i)}, \lambda_i)$
\newline
\newline
\newline
\newline


\color{blue}
\begin{equation*}
				A=T \Lambda T^{-1}, AT = \Lambda T 
\end{equation*}

\begin{equation*}
	\begin{aligned}
				A [t^{(1)}, t^{(2)}, \dots, t^{(n)}] = [t^{(1)}, t^{(2)}, \dots, t^{(n)}] * \begin{bmatrix}
\lambda_1 & 0 & \dots & 0 \\
0 & \lambda_2 & \dots & 0 \\
\vdots & \vdots & \ddots & \vdots \\
0 & 0 & \dots & \lambda_n \\
\end{bmatrix} \\
[At^{(1)}, At^{(2)}, \dots, At^{(n)}] = [\lambda_1 t^{(1)}, \lambda_2 t^{(2)}, \dots, \lambda_n t^{(n)}]
	\end{aligned}
\end{equation*}
\begin{equation*}
	At^{(i)} = \lambda_i t^{(i)}
\end{equation*}



%
\color{black}
\item[(b)] Let $A$ be symmetric. Show that if $U = \left[ u^{(1)} \ \cdots \ u^{(n)} \right]$ is orthogonal, where $u^{(i)} \in \mathbb{R}^n$ and $A = U \Lambda U^T$, then $u^{(i)}$ is an eigenvector of $A$ and $A u^{(i)} = \lambda_i u^{(i)}$, where $\Lambda = \text{diag}(\lambda_1, \ldots, \lambda_n)$.

\color{blue}

\begin{align*}
AU &= U \Lambda U^T U = U \Lambda \\
A \begin{bmatrix}
    u^{(1)} & u^{(2)} & \cdots & u^{(n)}
\end{bmatrix} &= \begin{bmatrix}
    u^{(1)} & u^{(2)} & \cdots & u^{(n)}
\end{bmatrix} \begin{bmatrix}
    \lambda_1 & 0 & \cdots & 0 \\
    0 & \lambda_2 & \cdots & 0 \\
    \vdots & \vdots & \ddots & \vdots \\
    0 & 0 & \cdots & \lambda_n
\end{bmatrix} \\
\begin{bmatrix}
    Au^{(1)} & Au^{(2)} & \cdots & Au^{(n)}
\end{bmatrix} &= \begin{bmatrix}
    \lambda_1 u^{(1)} & \lambda_2 u^{(2)} & \cdots & \lambda_n u^{(n)}
\end{bmatrix} \\
A u^{(i)} &= \lambda_i u^{(i)}
\end{align*}

\color{black}
\item[(c)] Show that if $A$ is PSD, then $\lambda_i(A) \geq 0$ for each $i$.
\color{blue}
\begin{equation*}
	\begin{aligned}
		At^{(i)} &= \lambda _i t ^ {(i)}\\
		(t^(i))^TAt^{(i)} &= \lambda _i  || t^{(i)} ||_2  = \lambda _i \geq 0
	\end{aligned}
\end{equation*}
\end{enumerate}


























\end{document}