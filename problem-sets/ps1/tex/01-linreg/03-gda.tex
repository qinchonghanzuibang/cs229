\clearpage
\item \subquestionpoints{5}
Recall that in GDA we model the joint distribution of $(x, y)$ by the following
equations:
%
\begin{eqnarray*}
	p(y) &=& \begin{cases}
	\phi & \mbox{if~} y = 1 \\
	1 - \phi & \mbox{if~} y = 0 \end{cases} \\
	p(x | y=0) &=& \frac{1}{(2\pi)^{n/2} |\Sigma|^{1/2}}
		\exp\left(-\frac{1}{2}(x-\mu_{0})^T \Sigma^{-1} (x-\mu_{0})\right) \\
	p(x | y=1) &=& \frac{1}{(2\pi)^{n/2} |\Sigma|^{1/2}}
		\exp\left(-\frac{1}{2}(x-\mu_1)^T \Sigma^{-1} (x-\mu_1) \right),
\end{eqnarray*}
%
where $\phi$, $\mu_0$, $\mu_1$, and $\Sigma$ are the parameters of our model.

Suppose we have already fit $\phi$, $\mu_0$, $\mu_1$, and $\Sigma$, and now
want to predict $y$ given a new point $x$. To show that GDA results in a
classifier that has a linear decision boundary, show the posterior distribution
can be written as
%
\begin{equation*}
	p(y = 1\mid x; \phi, \mu_0, \mu_1, \Sigma)
	= \frac{1}{1 + \exp(-(\theta^T x + \theta_0))},
\end{equation*}
%
where $\theta\in\Re^n$ and $\theta_{0}\in\Re$ are appropriate functions of
$\phi$, $\Sigma$, $\mu_0$, and $\mu_1$.

\ifnum\solutions=1{
  \newcommand\given[1][]{\:#1\vert\:}


\begin{answer}
%
%\begin{equation*}
%		p(y \mid x) = \frac{p(x \mid y)\cdot p(y)}{p(x)},
%		\text{ where }  p(x) = p(x \mid y =1) \cdot p(y=1) + p(y=0 \mid x) \cdot p(y=0) 
%\end{equation*}
%
%\begin{equation*}
%	\begin{aligned}
%				p(y=1 \mid x; \phi, \mu_0, \mu_1, \Sigma)  = \frac{p(x \mid y=0) \cdot p(y=1)}{p(x \mid y =1) \cdot p(y=1) + p(y=0 \mid x) \cdot p(y=0)} \\
%				= \frac{\frac{1}{(2\pi)^{n/2}|\Sigma|^{1/2}} \exp\left(-\frac{1}{2}(x - \mu_0)^T \Sigma^{-1} (x - \mu_0)\right) \cdot \phi}{\frac{1}{(2\pi)^{n/2}|\Sigma|^{1/2}} \exp\left(-\frac{1}{2}(x - \mu_1)^T \Sigma^{-1} (x - \mu_1)\right) \cdot \phi + \frac{1}{(2\pi)^{n/2}|\Sigma|^{1/2}} \exp\left(-\frac{1}{2}(x - \mu_0)^T \Sigma^{-1} (x - \mu_0)\right) \cdot (1-\phi)} 
%	\end{aligned}
%\end{equation*}

\begin{flalign*}
			& p(y \mid x) = \frac{p(x \mid y)\cdot p(y)}{p(x)},
		\text{ where }  p(x) = p(x \mid y =1) \cdot p(y=1) + p(y=0 \mid x) \cdot p(y=0) & \\ 
			& p(y=1 \mid x; \phi, \mu_0, \mu_1, \Sigma)  = \frac{p(x \mid y=1) \cdot p(y=1)}{p(x \mid y =1) \cdot p(y=1) + p(y=0 \mid x) \cdot p(y=0)} & \\
			& = \frac{\frac{1}{(2\pi)^{n/2}|\Sigma|^{1/2}} \exp\left(-\frac{1}{2}(x - \mu_1)^T \Sigma^{-1} (x - \mu_1)\right) \cdot \phi}{\frac{1}{(2\pi)^{n/2}|\Sigma|^{1/2}} \exp\left(-\frac{1}{2}(x - \mu_1)^T \Sigma^{-1} (x - \mu_1)\right) \cdot \phi + \frac{1}{(2\pi)^{n/2}|\Sigma|^{1/2}} \exp\left(-\frac{1}{2}(x - \mu_0)^T \Sigma^{-1} (x - \mu_0)\right) \cdot (1-\phi)}  & \\
			& = \frac{1}{1+\exp(\frac{1}{2}(x-\mu_1)^T \Sigma^{-1}(x-\mu_1)-\frac{1}{2}(x-\mu_0)^T \Sigma^{-1}(x-\mu_0))\cdot \frac{1-\phi}{\phi}}) & 	
\end{flalign*}


Expand the Quadratic Forms:
\[
\frac{1}{2} (x - \mu_1)^T \Sigma^{-1} (x - \mu_1) = \frac{1}{2} \left( x^T \Sigma^{-1} x - 2 x^T \Sigma^{-1} \mu_1 + \mu_1^T \Sigma^{-1} \mu_1 \right)
\]
\[
\frac{1}{2} (x - \mu_0)^T \Sigma^{-1} (x - \mu_0) = \frac{1}{2} \left( x^T \Sigma^{-1} x - 2 x^T \Sigma^{-1} \mu_0 + \mu_0^T \Sigma^{-1} \mu_0 \right)
\]

Subtract the Quadratic Forms:
\[
\frac{1}{2} \left( x^T \Sigma^{-1} x - 2 x^T \Sigma^{-1} \mu_1 + \mu_1^T \Sigma^{-1} \mu_1 \right) - \frac{1}{2} \left( x^T \Sigma^{-1} x - 2 x^T \Sigma^{-1} \mu_0 + \mu_0^T \Sigma^{-1} \mu_0 \right)
\]
Simplify to:
\[
\frac{1}{2} \left( - 2 x^T \Sigma^{-1} (\mu_1 - \mu_0) + \mu_1^T \Sigma^{-1} \mu_1 - \mu_0^T \Sigma^{-1} \mu_0 \right)
\]
Factor out the terms:
\[
- x^T \Sigma^{-1} (\mu_1 - \mu_0) + \frac{1}{2} (\mu_1^T \Sigma^{-1} \mu_1 - \mu_0^T \Sigma^{-1} \mu_0)
\]

Combine with the Remaining Term:
\[
- x^T \Sigma^{-1} (\mu_1 - \mu_0) + \frac{1}{2} (\mu_1^T \Sigma^{-1} \mu_1 - \mu_0^T \Sigma^{-1} \mu_0) + \ln \left( \frac{\phi}{1 - \phi} \right)
\]

Combine all terms:
\[
- x^T \Sigma^{-1} (\mu_1 - \mu_0) + \frac{1}{2} (\mu_0 + \mu_1)^T \Sigma^{-1} (\mu_0 - \mu_1) - \ln \left( \frac{1 - \phi}{\phi} \right)
\]

Final Expression:
\[
\frac{1}{1 + \exp \left\{ - \left( \Sigma^{-1} (\mu_1 - \mu_0) \right)^T x + \frac{1}{2} (\mu_0 + \mu_1)^T \Sigma^{-1} (\mu_0 - \mu_1) - \ln \left( \frac{1 - \phi}{\phi} \right) \right\} }
\]

Where: 
\[
\theta = \Sigma^{-1} (\mu_1 - \mu_0)
\]
\[
\theta_0 = \frac{1}{2} (\mu_0 + \mu_1)^T \Sigma^{-1} (\mu_0 - \mu_1) - \ln \left( \frac{1 - \phi}{\phi} \right)
\]
%\begin{equation*}
%\begin{aligned}
%		\frac{1}{2} (x - \mu_1)^T \Sigma^{-1} (x - \mu_1) = \frac{1}{2} \left( x^T \Sigma^{-1} x - 2 x^T \Sigma^{-1} \mu_1 + \mu_1^T \Sigma^{-1} \mu_1 \right) \\
%\frac{1}{2} (x - \mu_0)^T \Sigma^{-1} (x - \mu_0) = \frac{1}{2} \left( x^T \Sigma^{-1} x - 2 x^T \Sigma^{-1} \mu_0 + \mu_0^T \Sigma^{-1} \mu_0 \right)
%\end{aligned}
%\end{equation*}




\end{answer}

}\fi
